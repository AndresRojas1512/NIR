\chapter*{ЗАКЛЮЧЕНИЕ}
\addcontentsline{toc}{chapter}{ЗАКЛЮЧЕНИЕ}

Цель научно-исследовательской работы: изучение и анализ существующих функций активации свёрточных нейронных сетей и оценка их эффективности при использовании в различных приложениях. Цель достигнута. Для достижения этой цели были выполнены следующие задачи:

\begin{itemize}[label=---]
	\item описание и анализ предметной области: были введены основные понятия, связанные с функциями активации в контексте СНС;
	\item проведен обзор и классификация существующих функций активации;
	\item на основе установленных критериев было выполнено сравнение описанных функций активации.
\end{itemize}

Выбор функции активации может значительно повлиять на производительность СНС в зависимости от конкретной задачи и характеристик данных. Функции активации, такие как ELU или Mish, показали высокую эффективность в задачах, где важно учитывать отрицательные значения, в то время как классический ReLU оставался предпочтительным выбором для сценариев, требующих высокой вычислительной эффективности и простоты.

Результаты исследования данной работы могут быть использованы для оптимизации архитектур свёрточных нейронных сетей в специализированных приложениях, улучшая тем самым их способность к обучению и общую производительность.