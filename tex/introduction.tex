\chapter*{ВВЕДЕНИЕ}
\addcontentsline{toc}{chapter}{ВВЕДЕНИЕ}

Сверточные нейронные сети (далее СНС) -- вид нейронных сетей, которые применяются в области компьютерного зрения и анализа изображений. Они используют математическую операцию свертки, что позволяет выделять важные признаки из входных данных и обеспечивает высокую эффективность при обработке больших наборов данных \cite{understanding}.

Основные облясти применения СНС включают распознавание и классификацию изображений, видеоанализа, распознавание речи и аудионализа, а также обработку естественного языка. СНС используются в медицинских приложениях для анализа медицинских изрображений, в автономных транспортных средствах для обработки и интерпретации визуалной информации, а также в системах видеонаблюдения для обнаружения и отслеживания объектов \cite{survey}.

Актуальность развития сверточных нейронных сетей подтверждается их эффективностью в обработке наборов данных, полученных из изображений, видеозаписей.

Целью данной работы заключается в рассмотрении и описании архитектуры сверточных нейронных сетей.
Для достижения поставленной цели предстоит решить следующие задачи:

\begin{itemize}[label=---]
	\item описание основных понятий СНС;
	\item описание и анализ функций активации СНС;
	\item сравнение и оценка эффективности функций активации СНС.
\end{itemize}
