\chapter{Анализ предметной области}

\section{Основные определения}
 Сверточная нейронная сеть является управляемой нейронной сетью, базовая структура которой включает входной слой, свёрточные слои, слои пулинга (подвыборки), полносвязные слои и выходной слой \cite{recursive}.
 
 \imgScale{0.53}{cnn_base_config}{Классическая архитектура сверточной нейронной сети}
 
Основные функции каждой части СНС могут быть описаны следующим образом.
\subsection{Сверточный слой}
	Сверточный слой состоит из множества карт признаков (англ. feature maps), каждая из которых содержит множество нейронов. Каждый нейрон соединяется с локальной областью предыдущей карты признаков через сверточное ядро. Эти карты признаков позволяют сверточному слою определять выход нейронов нейронной сети. Нейронная сеть извлекает различные признаки входных данных с помощью сверточных операций. Эта специальная структура приводит к характеристике, называемой разделением весов. В одном и том же сверточном слое сверточное ядро с одинаковым весом применяется ко всем позициям данных. Нейроны внутри сверточного слоя соединяются только с небольшой областью предшествующего слоя. Таким образом, сверточному слою нужно изучать только один набор признаков, а затем продвигать повторное использование этих признаков для получения дополнительных абстрактных признаков в высокоуровневом выражении.
	Первые и последующие свёрточные слои извлекают признаки низкого и высокого уровня соответственно. Предположим, что вход сверточного слоя обозначен как \( a \). В сверточном слое активированный выход \( l \)-го слоя может быть выражен как
	\begin{equation}
		C^{l} = f(a^{l} \ast W^{l} + b^{l})
	\end{equation}
	где \( C^{l} \) --- это выход \( l \)-го слоя; \( a^{l} \) --- вход этого \( l \)-го сверточного слоя, который также является выходом предыдущего \( (l - 1) \)-го слоя; \( W^{l} \) --- вектор весов \( l \)-го сверточного слоя \( l \)-й слой; \( b^{l} \) --- вектор смещения \( (l - 1) \)-го слоя; \(\ast\) указывает на сверточную операцию; \( f() \) --- функция активации.
\subsection{Слой подвыборки}
	Слой подвыборки следует за сверточным слоем и дополнительно модифицирует выход слоя. Каждая характеристическая поверхность этого слоя соответствует слою фильтра сверточного слоя. Слой пулинга часто использует операцию проекцирования по пространственной размерности входного сигнала, тем самым дополнительно уменьшая количество параметров. Слой пулинга обычно использует операцию максимального пулинга, которая выводит максимальное значение входящих сигналов, тем самым выделяя наиболее значимые признаки. Процесс максимального пулинга уменьшает размерность второго этапа извлечения признаков, что снижает масштаб признаков, но сохраняет неизменным количество признаков \cite{lightweight}.
	
	После слоя пулинга, СНС может извлекать признаки высокого уровня. Пусть \( P \) будет выходом \( l \)-го слоя, который является слоем пулинга. Тогда,
	\begin{equation}
		P = \text{subsampling}(C^{(l-1)}) = \max(C^{(l-1)}).
	\end{equation}
	
	Свёрточные и слои пулинга могут чередоваться между собой несколько раз. Затем один или два слоя полносвязных нейронов применяются поверх них. Каждый нейрон в полносвязном слое напрямую связан со всеми нейронами в предыдущем слое. Эта полная связность является основной для архитектуры СНС. Следствие это для полной связности выполняют извлечение признаков, а полносвязный слой выравнивает признаки из матричной формы в одномерную последовательность. Выходное значение этого слоя используется для вывода на последний слой, который дополнительно классифицирует извлеченные признаки \cite{lightweight}.
	
\subsection{Функция активации}
	Функция активации --- механизм, который определяет, какая информация должна передаваться следующему нейрону. Каждый нейрон в нейронной сети принимает выходное значение нейронов предыдущего слоя в качестве входа и передает обработанное значение следующему слою. В многослойной нейронной сети между двумя слоями действует функция активация \cite{survey}.
	
	\imgScale{0.37}{activation_function}{Схема функции активации}
	
	где \(x_i\) --- входной признак; \(n\) --- количество признаков, которые поступают в нейрон \(j\) одновременно; \(w_{ij}\) --- весовой коэффициент связи между входным признаком \(x_i\) и нейроном \(j\); \(b_j\) --- внутреннее состояние нейрона \(j\), которое является значением смещения; \(y_j\) --- выход нейрона \(j\); \(f\) --- функция активации.


\section{Применение сверточных нейронных сетей в области компьютерного зреня}

Свёрточные нейронные сети (СНС) являются фундаментальным инструментом в области компьютерного зрения, обеспечивая эффективное распознавание и классификацию объектов на изображениях. Основная цель использования СНС заключается в улучшении точности и эффективности анализа изображений.

На рисунке \ref{img:idef0.png} представлена IDEF0-диаграмма верхнего уровня задачи.

\imgScale{0.55}{idef0.png}{IDEF0-Диаграмма фунциональной декомпозиции уровня A0}

	

